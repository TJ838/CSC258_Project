\documentclass{article}

%% Page Margins %%
\usepackage{geometry}
\geometry{
    top = 0.75in,
    bottom = 0.75in,
    right = 0.75in,
    left = 0.75in,
}

\usepackage{amsmath}
\usepackage{graphicx}
\usepackage{parskip}

\title{Assembly Project: Dr Mario}

% TODO: Enter your name
\author{Adeeb Nawshad & Taiyi Jin}

\begin{document}
\maketitle

\section{Instruction and Summary}

\begin{enumerate}

    \item Which milestones were implemented? 
    % TODO: List the milestone(s) and in the case of 
    %       Milestones 4 & 5, list what features you 
    %       implemented, sorted into easy and hard 
    %       categories.
    Milestone 1 to 5.
    We completed easy features 1,2,3,5,6,11 and hard feature 5

    \item How to view the game:
    % TODO: specify the pixes/unit, width and height of 
    %       your game, etc.  NOTE: list these details in
    %       the header of your breakout.asm file too!
    
    
    \begin{enumerate}

    \item
    Unit width and height in pixels:8
    \item
    Display width and height in pixels:256
    \item
    The left part is the playing field, the right part displays the next capsule


    \end{enumerate}

    

\begin{figure}[ht!]
    \centering
    \includegraphics[width=0.3\textwidth]{name.png}
    \caption{a simple view of the game}
    \label{Instructions}
\end{figure}

\item Game Summary:
% TODO: Tell us a little about your game.
\begin{itemize}
\item
The player selects from the three difficulties, which decides the speed of drop and the number of viruses in the game.
\item
The user gets to move the capsule down, left or right, and can rotate the capsule 360 degrees. When a capsule is dropped, a new capsule will be generated at the top, which is shown on the right of the screen.
\item 
The viruses are displayed in the play field with a slightly different color. The capsules will get cleared once we have 4 or more capsules in a column/row. When all viruses are cleared, the game ends. The game also ends when the entrance of the play field gets blocked.
\end{itemize}

    
\end{enumerate}

\section{Attribution Table}
% TODO: If you worked in partners, tell us who was 
%       responsible for which features. Some reweighting 
%       might be possible in cases where one group member
%       deserves extra credit for the work they put in.

\begin{center}
\begin{tabular}{|| c | c ||}
\hline
 Adeeb Nawshad, 1009812945 &  Taiyi Jin, 1009075796 \\ 
 \hline
 Milestone 1 & Milestone 2\\
 \hline
 Easy feature 1 & Milestone 3\\
 \hline
 Easy feature 2 & Easy feature 5\\ 
 \hline
 Easy feature 3 & Easy feature 11\\ 
 \hline
 Easy feature 6 & Hard feature 5\\ 
 \hline
\end{tabular}
\end{center}

% TODO: Fill out the remainder of the document as you see 
%       fit, including as much detail as you think 
%       necessary to better understand your code. 
%       You can add extra sections and subsections to 
%       help us understand why you deserve marks for 
%       features that were more challenging than they
%       might initially seem.


\end{document}